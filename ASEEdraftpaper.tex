\documentclass[12pt]{article}

\usepackage[english]{babel}
\usepackage[utf8x]{inputenc}
\usepackage{amsmath}
\usepackage{graphicx}
\usepackage{titling}
\usepackage[margin=1in]{geometry}
\usepackage{times}



%\setlength{\droptitle}{-10em}   % This is your set screw


\title{\large The Use of Student Generated Lab Activities to Enhance Engagement}
\author{\normalsize Devin R. Berg\\\normalsize bergdev@uwstout.edu\\\normalsize Engineering \& Technology Department\\\normalsize University of Wisconsin - Stout}
\date{}

\begin{document}

\maketitle



\begin{abstract}
The use of lab activities in a Mechanics of Materials course is fundamental to enhancing the student's understanding of the core concepts as well as providing a hands-on educational experience. However, the traditional lab activities (tension test, torsion test, etc.), while well founded, do little to generate excitement or engage the student in the learning process.

As a means improving student engagment in the lab activity, individual student groups were granted freedom in designing their own lab activity for the beam bending portion of the course under the guidelines that it must demonstrate the theoretical concepts presented in lecture in some way. Students were assessed based on relevance to course concepts and quality of the demonstration.

While there was much variety in approaches to this task, the majority of student demonstrations were well formed in the context of beam bending theory. Outcomes of this pilot experiment will be presented as well as suggestions for future revisions.
\end{abstract}



\end{document}

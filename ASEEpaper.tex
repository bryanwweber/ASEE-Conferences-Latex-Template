\documentclass[12pt]{article}
\usepackage[english]{babel}
\usepackage[utf8x]{inputenc}
\usepackage{amsmath}
\usepackage{graphicx}
\usepackage{etoolbox}
\usepackage{changepage}
\usepackage{titlesec}
\usepackage[parfill]{parskip}
\usepackage[margin=1in]{geometry}
\usepackage{times}
\usepackage{float}
\usepackage[numbers,super]{natbib}
\usepackage{lipsum} % Package to generate dummy text throughout this template. Remove for real use.

\titleformat*{\section}{\normalsize\bfseries} % Makes section titles 12 pt font


%----------------------------------------------------------------------------------------
%  TITLE SECTION
%----------------------------------------------------------------------------------------
\title{\large \textbf{This is where you put your super awesome title that describes your paper and wows the world}} % using \large makes the title approximately 14 pt.
% Author info isn't included for the Annual Conference but some regional conferences might request it.
\author{}
%\author{\normalsize Author Name\\
%\normalsize email@example.com\\
%\normalsize Name of Your Department\\\
%\normalsize Your Institution Name}
\date{} % This leaves the date blank.

\makeatletter % This gets the margins for the title set.
\patchcmd{\@maketitle}{\begin{center}}{\begin{adjustwidth}{0.5in}{0.5in}\begin{center}}{}{}
\patchcmd{\@maketitle}{\end{center}}{\end{center}\end{adjustwidth}}{}{}
\makeatother

%----------------------------------------------------------------------------------------

\begin{document}
\raggedright
\maketitle
\thispagestyle{empty}
\pagestyle{empty}

%----------------------------------------------------------------------------------------
%  PAPER CONTENTS
%----------------------------------------------------------------------------------------
\section*{Abstract}

\lipsum[1] % Dummy text

%------------------------------------------------

\section*{Introduction}

\lipsum[2-3] % Dummy text

%------------------------------------------------

\section*{Methods}

Maecenas sed ultricies felis. Sed imperdiet dictum arcu a egestas.
\begin{itemize}
\item Donec dolor arcu, rutrum id molestie in, viverra sed diam
\item Curabitur feugiat
\item turpis sed auctor facilisis
\item arcu eros accumsan lorem, at posuere mi diam sit amet tortor
\item Fusce fermentum, mi sit amet euismod rutrum
\item sem lorem molestie diam, iaculis aliquet sapien tortor non nisi
\item Pellentesque bibendum pretium aliquet
\end{itemize} 
\lipsum[4] % Dummy text

%------------------------------------------------

\section*{Results}

\begin{table}[H]
\caption{Example table}
\centering
\begin{tabular}{llr}
\multicolumn{2}{c}{Name} \\
First name & Last Name & Grade \\
John & Doe & $7.5$ \\
Richard & Miles & $2$ \\
\end{tabular}
\end{table}

\lipsum[5] % Dummy text

As seen in equation \ref{eq:emc} and proven by Einstein \cite{einstein1935can},
\begin{equation}
\label{eq:emc}
e = mc^2
\end{equation}

\lipsum[6] % Dummy text

%------------------------------------------------

\section*{Discussion}

\lipsum[7-8] % Dummy text

%----------------------------------------------------------------------------------------
%  REFERENCE LIST
%----------------------------------------------------------------------------------------
\vspace{4\baselineskip}\vspace{-\parskip} % Creaters proper 4 blank line spacing.
\footnotesize % Makes bibliography 10 pt font.
\bibliographystyle{unsrtnat} %Can use a different style as long as it is one which uses numbered references in the text.
\bibliography{ASEEpaper}

%----------------------------------------------------------------------------------------



\end{document}
